\problemname{LHC}

For your latest top-secret experiment, you will need a large quantity of Higgs
bosons. To obtain these elusive particles, you will need to build a large
hadron collider, a long circular tunnel that you can use to accelerate
particles and smash them into each other.

You already have access to an extensive network of tunnels, which is guaranteed
to be connected and free of cyclic paths. In other words, the existing tunnels
form a tree structure. This system can be represented by $N$ junctions, labelled
$1$ through $N$, connected by $N - 1$ tunnels, each of which connects two junctions.
Tunnels can be traversed in either direction (i.e., if there is a junction from
$a$ to $b$, that junction also goes from $b$ to $a$).

By adding exactly one tunnel to the network, you can create a cyclic path,
which you will use to build your large hadron collider. You wish to form the
longest possible collider in this way, where we define length as the number of
tunnels in a cycle.

For example, in the following network, we can form a collider of length 4 by
building a tunnel between junctions 1 and 5, or between 2 and 5:

\begin{center}
\includegraphics[width=0.25\textwidth]{ccc13s2p3}
\end{center}

\section*{Input}

The first line of each test case will contain $N$ ($3 \leq N \leq 3\cdot 10^5$), the number
of junctions. The next $N - 1$ lines will each contain two space-separated
integers $i$ and $j$, indicating that there is a tunnel between junctions $i$ and $j$ ($1 \leq i,j \leq N$).

\section*{Output}

The output should consist of a single line with one integer:
the length of the longest possible collider.

